%\part*{Lezione 08/03/2021}
\section{Decadimento $\gamma$}
Questo tipo di decadimento è simile a quello che accade sugli stati atomici; spesso, infatti, il nucleo figlio in un certo processo (per esempio nei decadimenti $\beta$, appena visti) si trova in uno stato eccitato, che decade nel fondamentale liberando un fotone (da cui il nome del decadimento):
$$\nuc{X}{A}{Z}{N}^* \to \nuc{X}{A}{Z}{N} + \gamma $$
L'ordine di grandezza delle energie in gioco è $0.1 \div 10$ MeV. Spesso questo tipo di decadimento viene usato per determinare il $J^\pi$ degli stati eccitati, dal momento che questo è legato a quello del fondamentale.\\
Anche se si ha la produzione di un fotone, alcuni processi in fisica sono dei decadimenti $\gamma$ \vir{camuffati}, ovvero non sono dei decadimenti ma processi di scattering, come per $n+p\to d +\gamma$

\paragraph{Energia dei livelli} Analizziamo a questo punto il processo di decadimento $\gamma$ dal punto di visto dell'equilibrio energetico.\\
Nel sistema del centro di massa:
$$\vec{P}_R + \vec{p}_\gamma = \vec{0}$$
dove abbiamo indicato con $\vec{P}_R$ l'impulso del nucleo dovuto al rinculo, per cui $T_R = P_R^2/2M$; poiché i fotoni hanno energia\footnote{Come al solito $c=1$} dell'ordine di 10 MeV (quindi $E_\gamma=p_\gamma = P_R$) siamo in regime non relativistico per il nucleo X.
\begin{displaymath}
\begin{aligned}
E_i&=E_f + E_\gamma + T_R \\
\Delta E &= E_\gamma + \frac{E_\gamma^2}{2M}
E_\gamma &= M\Bigl [ -1 + \sqrt{1+\frac{2\Delta E}{M}} \Bigr]
\end{aligned}
\end{displaymath}
dove abbiamo definito %\footnote{A priori $\Delta E \not =0$ perché potrei avere come stato finale un altro stato eccitato } 
$\Delta E \equiv E_i-E_f$ e abbiamo scartato la soluzione negativa perché priva di senso fisico. A questo punto poiché $\Delta E /M \ll 1$ sviluppiamo la radice:
\begin{displaymath}
\begin{aligned}
E_\gamma &\simeq M\Bigl [ -1 + 1+ \frac{\Delta E}{M} -\frac{1}{2}(\frac{\Delta E}{M})^2 + \dots \Bigr] \\
E_\gamma &\simeq \Delta E - \frac{\Delta E^2}{2M} \sim \Delta E
\end{aligned}
\end{displaymath}
Spesso anche il secondo ordine è trascurabile, per cui l'energia del fotone ci fornisce informazione sulla distanza tra i livelli di partenza e di arrivo.

\subsection{Teoria classica} A scopo illustrativo, studiamo dapprima il decadimento secondo una trattazione classica\footnote{In questa trattazione riprendiamo concetti classici di elettromagnetismo.}.\\
Se supponiamo di avere una distribuzione di cariche e correnti statiche, avremo allora anche un campo elettrico e un campo magnetico statici che possiamo riscrivere in termini di momenti di multipolo. Se tali cariche e correnti non sono statiche, sarà sempre possibile lo sviluppo, ma avremo campi di radiazione. A titolo di esempio consideriamo il dipolo\footnote{Le espressioni di dipolo elettrico e magnetico sono quelle per le configurazioni più semplici (cariche $+q$ e $-q$ a una certa distanza $z(t)$ e spira di area $A$ percorsa da una corrente $i(t)$).}:
\begin{displaymath}
\begin{aligned}
\text{\textbf{Dipolo} }&\text{\textbf{Elettrico}}\index{dipolo!elettrico} & \text{\textbf{Dipolo} }&\text{\textbf{Magnetico}}\index{dipolo!magnetico} \\
d(t) &= qz \cos{\omega t} & \mu(t)&= i A \cos{\omega t} \\
\vec{r}&\to-\vec{r} & \vec{r}&\to-\vec{r} \\
 \vec{E}\to&\vec{E};\;\; \vec{B}\to-\vec{B}\quad & \vec{E}\to-&\vec{E}; \;\; \vec{B}\to\vec{B} \\
\vec{B} &= k^2 (\vec{n}\times \vec{d}) \frac{e^{ikr}}{r} & \vec{B}&= ik\; \vec{n}\times \vec{A} \propto \vec{n} \times (\vec{n}\times \vec{\mu}) \\
\vec{E} &= \vec{B}\times \vec{n} & \vec{E} &= \vec{B}\times \vec{n} 
\end{aligned}
\end{displaymath}
Poiché l'interazione di particelle cariche è proporzionale a prodotto $\vec{v}\cdot\vec{A}$ e questo ha la stessa parità di $\vec{B}$, è il campo magnetico a determinare la parità del dipolo; si ha quindi: $\pi=-1$ per quello elettrico e $\pi=+1$ per quello magnetico.\\
Riassumiamo adesso alcuni concetti di Fisica III\footnote{Dal momento che sono solo rimandi, non ci soffermiamo molto sulle spiegazioni.}:
\begin{enumerate}
    \item La potenza irradiata in $dA$ a $\theta$ dall'asse $z$ è proporzionale al $\sin^2{\theta}$.
    \item La potenza irradiata media del dipolo è data dall'espressione:
    \begin{displaymath}
    \begin{aligned}
    P&= \frac{1}{12\pi\varepsilon_0} \frac{\omega^4}{c^3} \, d^2 & &\text{elettrico} \\
    P&= \frac{1}{12\pi\varepsilon_0} \frac{\omega^4}{c^5} \, \mu^2 & &\text{magnetico} 
    \end{aligned}
    \end{displaymath}
\end{enumerate}
\noindent Cerchiamo allora di generalizzarli introducendo una nuova notazione per lo sviluppo in multipolo attraverso l'indice $L$ in modo che $2^L$ sia l'ordine del multipolo (per esempio $L=1$ dipolo, $L=2$ quadrupolo,\dots) e l'indice $E$ o $M$ a seconda se lo sviluppo sia di multipolo elettrico o magnetico. Formalizzando si ha che la radiazione $2^L$ ha una distribuzione angolare proporzionale al \textbf{polinomio di Legendre}\index{polinomio di Legendre}\footnote{Si vede infatti che 
\begin{displaymath}
\begin{aligned}
L&=1 & \mathcal{P}_2(\cos{\theta})&= \frac{1}{2} (3\cos^2{\theta}-1)\sim\sin^2{\theta} \\
L&=2 & \mathcal{P}_4(\cos{\theta})&= \frac{1}{8} (35\cos^4{\theta}-30\cos^2{\theta}+3)
\end{aligned}
\end{displaymath}
che hanno l'andamento cercato.} di ordine $2L$ $\mathcal{P}_{2L}(\cos{\theta})$. È possibile esprimere anche la parità in funzione di tali indici:
$$\pi(ML)= (-)^{L+1} \qquad\quad \pi(EL)=(-)^L$$
Possiamo allora scrivere un'espressione\footnote{In questa espressione compare il doppio fattoriale che ricordiamo essere $(n)!! = n\cdot(n-2)\cdot\dots\cdot 1.$} per la potenza dello sviluppo:
$$P(\sigma L) = \frac{2(L+1)c}{\epsilon_0\,L[(2L+1)!!]^2} \bigl ( \frac{\omega}{c}\bigr)^{2L+2} \; [m(\sigma L)]^2 $$
con $\sigma = E,M$ e $m(\sigma L)$ \textbf{ampiezza del momento di multipolo}\index{ampiezza del momento di multipolo}\footnote{Nel caso più semplice $m(E1) \propto d$ e $m(B1)\propto \mu$.}.\\
A titolo di esempio, riportiamo il caso del dipolo elettrico:
$$P(E1) = \frac{4}{9\varepsilon_0}\frac{\omega^4}{c^3}\;[m(E1)]^2 \;\,\Rightarrow\;\, [m(E1)]^2 = \frac{3}{16\pi}\,d^2$$
Quest'ultimo risultato è molto utile in meccanica quantistica.

\subsection{Trattazione quantistica} Nell'ottica della meccanica quantistica abbiamo il passaggio da una funzione d'onda iniziale $\psi_i$ e una funzione d'onda finale $\psi_f$ dovuto agli operatori\footnote{In meccanica quantistica non si parla più di \textit{momenti}, ma di \textit{operatori}.} di multipolo. Per analogia con il decadimento $\beta$ ci aspettiamo che sia determinato da un elemento di matrice associato a un operatore\footnote{Nel caso del decadimento $\beta$ avevamo $M_{fi}$ e $O_X$.}:
$$m_{fi}(\sigma L) = \int \psi_f^*\, m(\sigma L)\, \psi_i d\Omega$$
Scriviamo il rate di emissione di un singolo fotone\footnote{Questo sarà la potenza irraggiata divisa per l'energia del singolo fotone.}:
\begin{equation*}\label{0308_rate}
    \lambda (\sigma L) = \frac{P(\sigma L)}{\hbar \omega} = \frac{2(L+1)}{\epsilon_0\hbar\,L[(2L+1)!!]^2} \bigl ( \frac{\omega}{c}\bigr)^{2L+1} \; |m(\sigma L)|^2
\end{equation*}
%$$$$
Per semplificare il calcolo facciamo queste due assunzioni: viene scambiato un singolo fotone e il processo coinvolge un singolo protone che passa da uno \textit{shell} a un altro.\\ Consideriamo inizialmente $\sigma = E$:
\begin{displaymath}
\begin{array}{ll}
    L=1 & d\to eZ \\
    L=2 & q\to 3Z^2\Omega^2
\end{array}
\end{displaymath}
Consideriamo una funzione d'onda a gradino, ovvero costante per $r<R$ e nulla altrimenti; allora dato l'operatore $e r^L Y_{LM}(\theta, \phi)$ l'elemento di matrice normalizzato per questa funzione d'onda è dato da:
$$m_{fi} \sim \frac{\int_0^R r^2 dr\: r^L}{\int_0^R r^2 dr} = \frac{r^{L+3}/ (L+3)}{r^3 / 3}\Biggr |_0^R = \frac{3}{L+3}R^L$$
dove non abbiamo considerato l'integrazione sulla parte angolare (l'armonica sferica). Sostituendo nel rate:
$$\lambda(EL)\simeq \frac{8\pi(L+1)}{L[(2L+1)!!]^2}\,\alpha\,\bigl(\frac{E}{\hbar c} \bigr)^{2L+1}\bigl(\frac{3}{L+3} \bigr)^2cR^{2L}$$
con $\alpha = e^2/4\pi\varepsilon_0\hbar c$ costante di struttura di fine\index{costante di struttura fine@$\alpha$-costante di struttura fine}.\\
Se $R=R_0A^{1/3}$ allora possiamo dare una stima dei vari termini di multipolo elettrico:
\begin{displaymath}
\begin{aligned}
\lambda(E1)&\simeq 10^{14} \, A^{2/3} E^3 \\
\lambda(E2)&\simeq 7\cdot10^{7} \, A^{4/3} E^5 \\
\lambda(E3)&\simeq 34 \, A^{2} E^7 \\
\lambda(E4)&\simeq 10^{-5} \, A^{8/3} E^9
\end{aligned}
\end{displaymath}
dove $E$ è in MeV (così da avere un rate in s$^{-1}$).\\
Consideriamo ora lo sviluppo per $\sigma =M$ con le stesse assunzioni fatte in precedenza:
$$m_{fi}\simeq \frac{3}{L+2}R^{L-1}$$
$$\lambda(ML) \simeq \frac{8\pi(L+1)}{L[(2L+1)!!]^2}\,\alpha\,\bigl(\frac{E}{\hbar c} \bigr)^{2L+1}\bigl(\frac{3}{L+2} \bigr)^2cR^{2L-2}\;\bigl(\frac{\hbar}{m_p c}\bigr)^2\Bigl ( \mu_p - \frac{1}{L+1} \Bigr )^2$$
Notiamo che l'espressione per il rate di multipolo magnetico non è molto diversa da quella per il multipolo elettrico, eccezion fatta per il termine in più che tiene conto del momento magnetico del protone (che non è quello di una particella elementare) e del $L+2$ a denominatore; tuttavia $L+2\sim L+3$ e il valore tipico per $(\mu_p-(L+1)^{-1})^2$ è circa 10, per cui possiamo stimare il rate di multipolo magnetico da quello di multipolo elettrico secondo:
$$\lambda (ML)\simeq \frac{10\hbar^2}{m_p^2c^2}\frac{\lambda(EL)}{R^2}$$
\begin{displaymath}
\begin{aligned}
\lambda(M1)&\simeq 3\cdot10^{13} \,  E^3 \\
\lambda(M2)&\simeq 2\cdot10^{7} \, A^{2/3} E^5 \\
\lambda(M3)&\simeq 10 \, A^{4/3} E^7 \\
\lambda(M4)&\simeq 3\cdot10^{-6} \, A^{2} E^9
\end{aligned}
\end{displaymath}
Queste stime per i rate dei termini di multipolo sono dette \textbf{stime di Weisskopf}\index{stime di Weisskopf}\label{sec-stime-Weiss} e sono state verificate sperimentalmente; esse permettono di dedurre che i termini con il contributo maggiore sono quelli a $L$ minori e che a parità di $L$ $\lambda(EL)>\lambda(ML)$.

\subsubsection{First order perturbation theory} Continuiamo lo sviluppo della teoria quantistica del decadimento secondo la così detta \textbf{first order perturbation theory}\index{first order perturbation theory}\footnote{Questo metodo è utilizzato anche per il calcolo dei fattori di forma.}.\\
Consideriamo l'hamiltoniana nel caso di interazione\footnote{Si tratta della sostituzione nell'hamiltoniana imperturbata $H_0 = p^2/2m$ di  $\vec{p}\to\vec{p}-e\vec{A}$ (\textit{minimal substitution}).} con il campo $\vec{A}$:
$$H = H_0 - \frac{e}{2m}(\vec{p}\cdot\vec{A}+\vec{A}\cdot\vec{p})+\frac{e^2}{2m}A^2$$
Il termine $e^2A^2/2m$ è quello dovuto alla transizione con 2 fotoni ed è molto meno significativo\footnote{Da qui in poi trascureremo questo contributo nell'hamiltoniana di interazione.} rispetto all'altro termine di interazione\index{hamiltoniana di interazione}\footnote{Si ricorda che $\vec{p}$ e $\vec{A}$ sono operatori con commutatore non nullo.}, che riscriviamo:
\begin{displaymath}
\begin{aligned}
(\vec{p}\cdot\vec{A}+\vec{A}\cdot\vec{p}) \,\psi &= -i\nabla_i (A_i\psi) - i A_i \nabla_i \psi = \\
&= -i (\nabla_i A_i )\psi - 2i A_i (\nabla_i\psi) = \\
&= -2i A_i \nabla_i \psi
\end{aligned}
\end{displaymath}
dove ci siamo messi nella \textbf{gauge di Coulomb}\index{gauge di Coulomb} ($\vec{\nabla}\cdot\vec{A}=0$). Riscrivendo l'hamiltoniana di interazione\index{hamiltoniana di interazione}:
$$H_I \simeq - \frac{e}{2m}\,(-2iA_i\nabla_i) = -\frac{e}{m}\, \vec{A}\cdot\vec{p}\:\Rightarrow\: -e \int d^3x \; \vec{A}(\vec{x})\cdot \vec{J}(\vec{x})$$
dove abbiamo riespresso il tutto in termini di densità di hamiltoniana, con $\vec{J}(\vec{x})$ densità di corrente nucleare\index{densità di corrente nucleare}\footnote{In questa espressione introduciamo l'indice $N$ per ricordare che si tratta della corrente nucleare.}, che per ora\footnote{Sono presenti infatti altri contributi che ricaveremo nel seguito.} introduciamo come:
$$\vec{J}_N (\vec{x}) = \sum_{i=1}^Z \frac{1}{2m}\,\Biggl \{ \delta^3(\vec{x}-\vec{r}_i),\, \vec{p}_i \Biggr \}$$
dove abbiamo indicato con $\bigl\{\:\bigr\}$ l'\textbf{anticommutatore}\index{anticommutatore}\footnote{L'anticommutatore di 2 operatori $A$ e $B$ è definito come:
$$\Bigl\{A,B\Bigr\} = AB+BA$$}. Scriviamo l'espressione esplicita anche per il campo esterno:
$$\vec{A}(\vec{x}) = \sum_k \frac{1}{\sqrt{2\omega_k\Omega}}\, \Bigl [ \underbrace{\widehat{\varepsilon}_{\vec{k}\lambda} a_{\vec{k}\lambda}\, e^{i\vec{k}\cdot\vec{x}}}_\text{Termine di assorbimento} + \underbrace{\widehat{\varepsilon}_{\vec{k}\lambda}^* a_{\vec{k}\lambda}^+\, e^{-i\vec{k}\cdot\vec{x}}}_\text{Termine di emissione}  \Bigr ]$$
dove abbiamo indicato con $a$ e $a^+$ gli operatori\footnote{Ricordiamo che per l'emissione e l'assorbimento $[a_{\vec{k}\lambda},a_{\vec{k}'\lambda'}^+] = \delta_{\vec{k}\vec{k}'}\delta_{\lambda\lambda'}$ e $a^+_{\vec{k}\lambda} \st{0} = \st{\vec{k}\lambda}$.} di \textit{distruzione}\index{operatore di distruzione} e \textit{creazione}\index{operatore di creazione} e con $\widehat{\varepsilon}$ il vettore di polarizzazione, definito come:
\begin{displaymath}
\begin{aligned}
\widehat{\varepsilon}_{\vec{k}0} &= \widehat{k} \\
\widehat{\varepsilon}_{\vec{k}\lambda\pm 1} &= \mp \frac{\widehat{\varepsilon}_{\vec{k}_x}\pm i\widehat{\varepsilon}_{\vec{k}_y}}{\sqrt{2}}
\end{aligned}
\end{displaymath}
Poiché stiamo studiando il decadimento $\gamma$ ci interessa solo il termine di emissione in $\vec{A}(\vec{x})$, allora:
\begin{displaymath}
\begin{aligned}
\oss{\vec{k}\lambda}{\vec{A}(\vec{x})}{0} &= \frac{1}{\sqrt{2\omega_k\Omega}} \widehat{\varepsilon}^*_{\vec{k}\lambda} e^{-i\vec{k}\cdot\vec{x}}\qquad \text{Funzione d'onda del fotone} \\
%
\oss{J_f M_f,\vec{k}\lambda}{H_I}{J_i M_i} &= - \frac{e}{\sqrt{2\omega_k\Omega}} \, \oss{J_f M_f}{\int d^3x \; e^{-i\vec{k}\cdot\vec{x}}\widehat{\varepsilon}^*_{\vec{k}\lambda}\cdot\vec{J}_N (\vec{x})}{J_i M_i}
\end{aligned}
\end{displaymath}
dove $J$ e $M$ sono riferiti al momento angolare totale del nucleo prima e dopo il decadimento.
Si osserva che la funzione d'onda del fotone si porta dietro un momento angolare\footnote{Ci aspetteremo allora delle regole di selezione}; infatti l'esponenziale può essere sviluppato in armoniche sferiche\index{armoniche sferiche} $\mathcal{Y}_{\ell m}(\hat{k})$:
\begin{displaymath}
\begin{aligned}
\widehat{\varepsilon}_{\vec{k}\lambda}\, e^{i\vec{k}\cdot\vec{x}} &= \widehat{\varepsilon}_{\vec{k}\lambda} \Biggl [ \sum_{\ell m} i^\ell (4\pi) \mathcal{Y}_{\ell m}^*(\hat{k})\mathcal{Y}_{\ell m}(\hat{x})\, \mathrm{j}_\ell (kx) \Biggr ] \\
%
\text{Se } \hat{k} \parallel \vec{z} \:\Rightarrow&\: \mathcal{Y}_{\ell m}^* (\hat{k}) = \mathcal{Y}_{\ell m} (\theta=0,\phi) = \sqrt{\frac{2\ell+1}{4\pi}} \delta_{m0} \\
%
\widehat{\varepsilon}_{\vec{k}\lambda}\, e^{i\vec{k}\cdot\vec{x}} &= \widehat{\varepsilon}_{\vec{k}\lambda} \sum_{\ell=0}^\infty i^\ell \sqrt{4\pi(2\ell+1)} \mathcal{Y}_{\ell 0}(\hat{x})\, \mathrm{j}_\ell (kx)
\end{aligned}
\end{displaymath}
Per arrivare a una forma più compatta dell'espressione, introduciamo le armoniche sferiche vettoriali\index{armoniche sferiche!vettoriali} $\vec{\mathcal{Y}}^M_{J\ell s}$:
\begin{displaymath}
\begin{aligned}
\vec{\mathcal{Y}}^M_{J\ell 1} &\equiv \sum_{m\lambda} \clebs{\ell m ,\, 1\lambda}{JM}\: \mathcal{Y}_{\ell m} (\hat{x}) \widehat{\varepsilon}_{1\lambda} \\
\mathcal{Y}_{\ell m} (\hat{x}) \widehat{\varepsilon}_{1\lambda} &= \sum_{JM} \clebs{\ell m ,\, 1\lambda}{JM} \: \vec{\mathcal{Y}}^M_{J\ell 1} \\
\Rightarrow\quad \widehat{\varepsilon}_{\vec{k}\lambda}\, e^{i\vec{k}\cdot\vec{x}} &= \widehat{\varepsilon}_{\vec{k}\lambda} \sum_{\ell=0}^\infty i^\ell \sqrt{4\pi(2\ell+1)}\, \mathrm{j}_\ell (kx) \, \sum_J \clebs{\ell 0 ,\, 1\lambda}{J\lambda}\, \vec{\mathcal{Y}}^{M=\lambda}_{J\ell 1}
\end{aligned}
\end{displaymath}
dove abbiamo introdotto le funzioni di Bessel $\mathrm{j}$\index{funzioni di Bessel@$\mathrm{j}$ funzioni di Bessel} e nel secondo passaggio abbiamo usato l'ortogonalità dei coefficienti di Clebsch-Gordan\index{coefficienti di Clebsch-Gordan}. Sempre per le proprietà di tali coefficienti abbiamo che $\clebs{\ell 0,\, 1\lambda}{J\lambda}$ è non nullo solo se $\ell = J,\,J-1,\,J+1$ e quindi possiamo riscrivere:
\begin{displaymath}
\begin{aligned}
\widehat{\varepsilon}_{\vec{k}\lambda}\, e^{i\vec{k}\cdot\vec{x}} = \sum_{J \geq 1} i^J \sqrt{\frac{4\pi (2J+1)}{2}} \;\Biggl \{& -\lambda \,\mathrm{j}_J (kx) \, \vec{\mathcal{Y}}^\lambda_{JJ1}\; + \\
&-i\Biggl [\sqrt{\frac{J+1}{2J+1}} \,\mathrm{j}_{J-1} (kx) \, \vec{\mathcal{Y}}^\lambda_{J(J-1)1}\; + \\
& -\sqrt{\frac{J}{2J+1}} \,\mathrm{j}_{J+1} (kx) \, \vec{\mathcal{Y}}^\lambda_{J(J+1)1} \Biggr ] \Biggr \} \\
%
\widehat{\varepsilon}_{\vec{k}\lambda}\, e^{i\vec{k}\cdot\vec{x}} = \sum_{J \geq 1} i^J \sqrt{\frac{4\pi (2J+1)}{2}} \;\Biggl \{&-\lambda\,\mathrm{j}_J (kx) \, \vec{\mathcal{Y}}^\lambda_{JJ1} - \frac{1}{k} \vec{\nabla}\land [\mathrm{j}_J (kx)\, \vec{\mathcal{Y}}^\lambda_{JJ1}]\Biggr \}
\end{aligned}
\end{displaymath}
dove nell'ultimo passaggio abbiamo usato le proprietà del prodotto $\vec{\nabla}\land \mathrm{j}\vec{\mathcal{Y}}$. Dunque il versore polarizzazione del fotone si porta dietro una parte proporzionale a un'armonica sferica vettoriale e una parte proporzionale al suo rotore (per questa ragione l'armonica sferica vettoriale è la \textit{funzione tipo} del fotone). A questo punto riscriviamo l'elemento di matrice\footnote{Nei calcoli ometteremo $\clebs{J_f M_f}{J_i M_i}$ per semplicità.} da cui eravamo partiti:
\begin{displaymath}
\begin{aligned}
\oss{J_f M_f,\vec{k}\lambda}{H_I}{J_i M_i} &= e \sum_{J\geq 1} (-i)^J \sqrt{\frac{2\pi(2J+1)}{2\omega_k\Omega}} \, \Bigl [ E_{J\,-\lambda} (k) + \lambda M_{J\,-\lambda} (k) \Bigr ] \\
E_{JM}(k) &= \frac{1}{k}\int d^3x\: [\vec{\nabla}\land \mathrm{j}_J (kx)\, \vec{\mathcal{Y}}^M_{JJ1}]\cdot \vec{J}(\vec{x}) \\
M_{JM}(k) &= \int d^3x\: [\mathrm{j}_J (kx)\, \vec{\mathcal{Y}}^M_{JJ1}]\cdot \vec{J}(\vec{x}) \\
\end{aligned}
\end{displaymath}
dove abbiamo introdotto i termini di multipolo elettrico $E_{JM}$\index{termini di multipolo!elettrico} e magnetico $M_{JM}$\index{termini di multipolo!magnetico}; infatti ricordandosi che $\vec{\mathcal{Y}}^{\lambda\,+}_{JJ1} = (-)^{\lambda+1}\,\vec{\mathcal{Y}}^{-\lambda}_{JJ1}$ si ha che la parità è  data da:
\begin{displaymath}
\begin{array}{l}
    \pi(EJ) = (-)^{J+1} \\
    \pi(MJ) = (-)^J  
\end{array}
\end{displaymath}
che è la parità che avevamo trovato anche nel conto classico.

\paragraph{Corrente di magnetizzazione} Avevamo anticipato che l'espressione della corrente nucleare non era completa, mancava infatti il contributo magnetico:
$$\vec{J}_N(\vec{x}) = \sum_{i=1}^Z \vec{J}_C(\vec{x}_i) + \sum_{i=1}^A \vec{J}_M (\vec{x}_i)$$
Vediamo allora da dove deriva questa corrente. Scriviamo l'operatore di momento magnetico\index{operatore di momento magnetico} del singolo nucleone:
\begin{displaymath}
\begin{aligned}
\mu_i (\vec{x}) &= \Bigl ( \mu_p \frac{1+\tau_z(i)}{2} + \mu_n \frac{1-\tau_z(i)}{2} \Bigr )\, \vec{\sigma}_i \mu_N \, \delta^3(\vec{x}-\vec{r}_i) \\
\mu_N &= \frac{e\hbar}{2m_pc} \\
\mu_p &= \frac{g_p}{2} = 2.79 \\
\mu_n &= \frac{g_n}{2} = -1.91
\end{aligned}
\end{displaymath}
Allora se consideriamo l'hamiltoniana\footnote{Questa corrisponde a $H = -\sum_i \vec{\mu}_i \cdot \vec{B}$, ma in termini di densità di hamiltoniana abbiamo rimosso gli indici e sostituito la somma con un integrale.} di interazione con il momento magnetico:
$$H = -\int d^3x \; \vec{\mu}\cdot(\vec{\nabla}\land \vec{A}) = -\int d^3x \; (\vec{\nabla}\land\vec{\mu})\cdot \vec{A} $$
$$\vec{J}_M (\vec{x}_i) \equiv \vec{\nabla}\land\vec{\mu}$$
che è appunto una \textbf{corrente di magnetizzazione}\index{corrente di magnetizzazione}.