%% COMPLEMENTI %%

\chapter{Complementi}

\section{Sul numero di neutrini}\label{compl-neutrini}
Come spiegato in \secrif{sec-nu-mass}, quando $T_e \to Q$ il neutrino non è più relativistico per cui $T_\nu\sim q^2/2m_\nu$ e $dq/dE_\nu = m_\nu/q$. Abbiamo allora che:
$$N(p)\propto p^2 \: \sqrt{Q-\sqrt{p^2+m_e^2} + m_e}$$ 
$$N(T_e)\propto \sqrt{T_e^2+2T_em_e} \: \sqrt{Q-T_e} \: (T_e+m_e)$$
Si nota che quando $T_e\to Q$ $dN/dp\to 0$ se $m_\nu = 0$ e $dN/dp\to \infty$ se $m_\nu \not = 0$, come si osserva in Figura \ref{0304_nu}; dalla pendenza è quindi possibile studiare il limite per la massa del neutrino.