%% COMPLEMENTI %%

\chapter{Approfondimenti}\label{complementi}

In questo capitolo ho raccolto alcuni argomenti che ho personalmente approfondito o testi e articoli che ho usato durante la preparazione dell'esame 

\section{Modelli e decadimenti}\label{compl-krane}
Una trattazione che riprende quella adottata in questi appunti riguardo ai modelli nucleari e i decadimenti $\beta$ e $\gamma$ si trova in Krane, K., S., \textit{\vir{Introductory Nuclear Physics}}, USA, John Wiley \& Sons, 1988.
\subsection{Decadimento $\varepsilon$}\label{compl-epsilon}
Nel calcolo del $Q$-value in \secrif{sec-qvalue} per la cattura $\varepsilon$ compare a sottrarre l'energia di legame del'elettrone catturato nell'$n$-esimo shell $B_n$, dove $n$ raccoglie tutti i numeri quantici che identificano tale particella. La ragione della presenza di questo termine è ben spiegata in (Krane, 1988), per cui ne riportiamo un piccolo riassunto: subito dopo la cattura il nucleo prodotto si trova in uno stato eccitato, per cui se l'elettrone occupava lo shell (interno) $k$ (identificato da $n=k,L,\dots$) il \vir{vuoto} da esso lasciato verrà subito \vir{riempito} da un elettrone di shell superiore; nel diseccitarsi l'elettrone emette un fotone di energia pari all'energia di legame $B_n$, che viene quindi persa.

\section{Regola d'oro di Fermi}\label{compl-orofermi}
Nei calcoli del rate molto spesso dall'espressione $\lambda$ si passa a quella differenziata $d\lambda$. Si tratta di una questione puramente formale e si rimanda alla lettura del paragrafo \textbf{Commenti sulla regola di Fermi} della sezione 20.2 \textit{Transizioni nel continuo e rate di un processo} delle dispense del professor G. Paffuti, \textit{Note di Meccanica Quantistica, anno accademico: 2017-2018}.\\
Tuttavia, per avere un'idea\footnote{Questa è una spiegazione sicuramente non rigorosa e particolarmente lacunosa, si consiglia la lettura del testo.} della giustificazione, si consideri l'espressione in \secrif{sec-teofermi}:
$$\lambda = \frac{2\pi}{\hbar}g^2 \Bigl |\int \phi_f^* O_X \phi_i d\Omega\Bigr |^2 \frac{p^2 dp 4\pi}{h^3}\frac{q^2dq4\pi}{h^3}\frac{1}{dE_f} \sim  p^2 q^2 dp $$
dove abbiamo fissato l'impulso $p$ dell'elettrone per cui $dE_f = dq$. Sarebbe quindi più corretto indicare $\lambda \to \lambda_p$ poiché è definito per quel particolare valore dell'energia dell'elettrone. Il rate totale sarà invece dato dalla somma su tutti gli impulsi possibili dell'elettrone finale; dal momento che l'impulso è una variabile continua si ha $\lambda_p \to d\lambda$. In altre parole, nella scrittura del rate si sottintende l'integrazione su $p$ per poi esplicitarla al momento opportuno.
 
\section{Sul numero di neutrini}\label{compl-neutrini}
Come spiegato in \secrif{sec-nu-mass}, quando $T_e \to Q$ il neutrino non è più relativistico per cui $T_\nu\sim q^2/2m_\nu$ e $dq/dE_\nu = m_\nu/q$. Abbiamo allora che:
$$N(p)\propto p^2 \: \sqrt{Q-\sqrt{p^2+m_e^2} + m_e}$$ 
$$N(T_e)\propto \sqrt{T_e^2+2T_em_e} \: \sqrt{Q-T_e} \: (T_e+m_e)$$
Si nota che quando $T_e\to Q$ $dN/dp\to 0$ se $m_\nu = 0$ e $dN/dp\to \infty$ se $m_\nu \not = 0$, come si osserva in Figura \ref{0304_nu}; dalla pendenza è quindi possibile studiare il limite per la massa del neutrino.

\section{Dettagli sul calcolo del decadimento $\gamma$}\label{compl-passaggi}
Nello sviluppo dell'onda piana in armoniche vettoriali in \secrif{sec-first-order} si ottiene l'espressione:
$$\widehat{\varepsilon}_{\vec{k}\lambda}\, e^{i\vec{k}\cdot\vec{x}} = \sum_{\ell=0}^\infty i^\ell \sqrt{4\pi(2\ell+1)}\, \mathrm{j}_\ell (kx) \, \sum_J \clebs{\ell 0 ,\, 1\lambda}{J\lambda}\, \vec{\mathcal{Y}}^{M=\lambda}_{J\ell 1}$$
\subsection{\CG}
Poiché $\ell = J, J\pm 1$, i coefficienti di \CG{} diversi da zero sono\footnote{Si ottengono calcolando $\clebs{\ell 0, 1 1}{J1}$ e poi usando le loro proprietà per passare a $\lambda = -1$.}:
\begin{align*}
	&\clebs{J0,1\lambda}{J\lambda} = - \frac{\lambda}{\sqrt{2}}\\
	&\clebs{J+1\,0,1\lambda}{J\lambda} = \frac{1}{\sqrt{2}} \sqrt{\frac{J}{2J+3}}\\
	&\clebs{J-1\,0,1\lambda}{J\lambda} = \frac{1}{\sqrt{2}} \sqrt{\frac{J+1}{2J-1}}
\end{align*}
\subsection{Proprietà della Bessel}
A questo punto si utilizza la proprietà della Bessel:
\begin{align*}
	\grad\land\mathrm{j}_J\vec{\mathcal{Y}}_{JJ1}^\lambda &= ik \Biggl [ \overbrace{\ppc{\frac{d}{d(kx)} - \frac{J}{kx}} \mathrm{j}_{J}(kx) }^{\mathrm{j}_{J+1}(kx)}\, \sqrt{\frac{J}{2J+1}}\, \vec{\mathcal{Y}}_{J,J+1,1}^\lambda + \\
	&+ \underbrace{\ppc{\frac{d}{d(kx)} - \frac{J+1}{kx}} \mathrm{j}_{J}(kx) }_{\mathrm{j}_{J-1}(kx)}\, \sqrt{\frac{J+1}{2J+1}}\, \vec{\mathcal{Y}}_{J,J-1,1}^\lambda \Biggr ]
\end{align*}
\subsection{Proprietà armonica vettoriale}\label{compl-passaggi-armonica}
Deriviamo la proprietà\footnote{Ricordarsi che $\widehat{\varepsilon}_{\vec{k}\lambda}^* = (-)^\lambda \; \widehat{\varepsilon}_{\vec{k},-\lambda}$ e che $\clebs{\ell m, S S_z}{J M} = (-)^{\ell+S-J} \clebs{\ell\, -m, S\, -S_z}{J \,-M} $}:
$$(\vec{\mathcal{Y}}_{JJ1}^\lambda)^* = (-)^{\lambda+1}\,\vec{\mathcal{Y}}^{-\lambda}_{JJ1}$$
\begin{align*}
	(\vec{\mathcal{Y}}_{JJ1}^\lambda)^* &= \clebs{J0,1\lambda}{J\lambda} \, \mathcal{Y}_{J0}^* \, \widehat{\varepsilon}_{\vec{k}\lambda}^* = \\
	&= (-)^\lambda \: \clebs{J0,1\lambda}{J\lambda} \, \mathcal{Y}_{J0} \, \widehat{\varepsilon}_{\vec{k},-\lambda} = \\
	&= (-)^{\lambda+1} \: \underbrace{\clebs{J0,1\,-\lambda}{J\,-\lambda} \, \mathcal{Y}_{J0} \, \widehat{\varepsilon}_{\vec{k},-\lambda}}_{\vec{\mathcal{Y}}_{JJ1}^{-\lambda}}
\end{align*}
Nel caso $|\lambda|=1$ allora $(\vec{\mathcal{Y}}_{JJ1}^\lambda)^* = \vec{\mathcal{Y}}_{JJ1}^{-\lambda}$.

\section{Effetto tunnel}\label{compl-tunnel}
Dimostriamo come ottenere l'espressione della probabilità di attraversamento per effetto tunnel nella sezione \secrif{sec-tunnel}.
\begin{align*}
	P &\sim \exp\PPg{-\frac{2}{\hbar}\int_0^a \sqrt{2m (V-E)}} = \\
	&= -\frac{2}{\hbar}\int_0^1 \sqrt{2mE}\frac{Z_1Z_2e^2}{E}\sqrt{\frac{1}{\rho}-1} \, d\rho = \\
	&= -2\pi \,\sqrt{\frac{m}{2E}} \, \frac{Z_1Z_2e^2}{\hbar}	
\end{align*} 
dove $\rho \equiv rE/Z_1Z_2e^2$ e per risolvere abbiamo usato $\rho = \sin^2(\alpha)$.


\section{CRM: indipendeza dal parametro $B$}\label{compl-CRM-dim}
Nel discutere il metodo della \textit{Calculable} $R$ MATRIX (sezione \secrif{sec-RM-C}) siamo arrivati all'espressione:
$$U_\ell = e^{2i\Phi_\ell} \, \frac{1+B\,R_\ell (E,B)-L_\ell^* \, R_\ell (E,B)}{1+B\,R_\ell(E,B)-L_\ell  \, R_\ell (E,B)}$$
Si può dimostrare\footnote{Seguiamo la dimostrazione dell'articolo Descouvemont, P. \& Baye, D., Rep. Prog. Phys., 2010, vol.3, \texttt{DOI:} \doi{10.1088/0034-4885/73/3/036301}, \texttt{arXiv:} \url{https://arxiv.org/abs/1001.0678}.\articolo{Descouvemont \& Baye}} che $U_\ell$ non dipende da $B$.\\
Consideriamo una matrice invertibile $\mathbf{V}$ di dimensione  $N\times N$ e due vettori di dimensione $N$ $\mathbf{u}$ e $\mathbf{v}$; possiamo sempre costruire una matrice quadrata $\mathbf{W}$ definita come:
$$\mathbf{W} = \mathbf{V} + \mathbf{u}\mathbf{v}^T$$
Notiamo che data l'espressione di $\mathbf{W}$ si ha:
$$\mathbf{W}^{-1} = \mathbf{V}^{-1} - \frac{\mathbf{V}^{-1}\mathbf{u}\mathbf{v}^T\mathbf{V}^{-1}}{1+\mathbf{v}^T\mathbf{V}^{-1}\mathbf{u}}$$
$$\mathbf{W}^{-1}\mathbf{u} = \frac{\mathbf{V}^{-1}\mathbf{u}}{1+\mathbf{v}^T\mathbf{V}^{-1}\mathbf{u}} \: \Rightarrow \: (\mathbf{v}^T\mathbf{W}^{-1}\mathbf{u})^{-1} = 1 + (\mathbf{v}^T\mathbf{V}^{-1}\mathbf{u})^{-1} $$
Osserviamo che se le matrici $\mathbf{V}\equiv C(E,B)$ e $\mathbf{W}\equiv C(E,0)$ e i vettori\footnote{Dove abbiamo preso $\ket{\varphi} \to \bm{\varphi}(a) = \ppc{\varphi_1 (a),\dots,\varphi_N(a)} = \sum_k \ket{\varphi_k}$; dunque $\scalar{\varphi_k}{\bm{\varphi}}=\scalar{\varphi_k}{\varphi_k}=1$.} $\mathbf{u} \equiv \ket{\varphi}$ e $\mathbf{v} \equiv \ket{\varphi}$ allora possiamo scrivere:
$$C(E,0) = C(E,B) + \frac{\hbar^2}{2\mu a} B\: \ket{\varphi}\bra{\varphi}$$
$$\ppc{\frac{\hbar^2}{2\mu a}}^{-1}(\bra{\bm{\varphi}}C(E,0)^{-1}\ket{\bm{\varphi}})^{-1} = B + \ppc{\frac{\hbar^2}{2\mu a}}^{-1}(\bra{\bm{\varphi}}C(E,B)^{-1}\ket{\bm{\varphi}})^{-1} $$
Usando l'espressione\footnote{Notare che $\bra{\bm{\varphi}}\mathbf{A}\ket{\bm{\varphi}} = \sum_{ij} \varphi_j A_{ij} \varphi_i$.} di $R_\ell$ in \eqref{0412_Rl2} si ottiene:
$$\frac{1}{R_\ell (E,0)} = B+ \frac{1}{R_\ell (E,B)}$$
dalla quale si ha l'indipendeza di $U_\ell$ cercata.

\section{Multipolarità di $\ce{^{14}N}(p,\gamma)\ce{^{15}O}$}\label{compl-multipoli}
Per l'analisi dei multipoli e altri approfondimenti sulla misura del fattore astrofisico della $\ce{^{14}N}(p,\gamma)\ce{^{15}O}$ (trattata nella sezione \secrif{sec-reaz-lenta}) consultare gli articoli Runkle, R.C., et al., Phys. Rev. Lett., 2005, vol.94,\texttt{DOI:}\doi{10.1103/PhysRevLett.94.082503}\articolo{Runkle et al.} e Formicola, A., et al., Phys. Lett. B, 2004, vol.591,\texttt{DOI:}\doi{10.1016/j.physletb.2004.03.092}\articolo{Formicola et al.}.